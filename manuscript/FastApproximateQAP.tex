\documentclass[11pt]{article}
\input{/Users/jovo/Research/other/latex/latex_document}

\title{\vspace{-75pt}(Brain) Graph Matching via \\ Fast Approximate Quadratic Programming}
\author{JoVo, cep}

\begin{document}
\maketitle


% \author{Joshua T.~Vogelstein, John M.~Conroy, Louis J.~Podrazik, Steven G.~Kratzer, Eric T.~Harley,
%         Donniell E.~Fishkind, 
% 		R.~Jacob~Vogelstein,
%         and~Carey~E.~Priebe% <-this % stops a space
% \IEEEcompsocitemizethanks{\IEEEcompsocthanksitem J.T. Vogelstein, E.T. Harley, D.E. Fishkind, and C.E. Priebe are with the Department
% of Applied Mathematics and Statistics, Johns Hopkins University, Baltimore, MD 21218. 
% %\protect\\
% % note need leading \protect in front of \\ to get a newline within \thanks as
% % \\ is fragile and will error, could use \hfil\break instead.
% E-mail: \{joshuav,eric.harley,def,cep\}@jhu.edu, \{conroyjohnm,ljpodra,sgkratz\}@gmail.com, jacob.vogelstein@jhuapl.edu
% \IEEEcompsocthanksitem J.M. Conroy, L.J. Podrazik and S.G. Kratzer are with Institute for Defense Analyses, Center for Computing Sciences, Bowie, MD 20708.
% \IEEEcompsocthanksitem R.J. Vogelstein is with the Johns Hopkins University Applied Physics Laboratory, Laurel, MD, 20723.}% <-this % stops a space
% \thanks{This work was partially supported by the Research Program in Applied Neuroscience.}}
 
\begin{abstract}
	
Graph matching (GM)---the process of finding an optimal permutation of the vertices of one graph to minimize adjacency disagreements with the vertices of another---is rapidly becoming an increasingly important computational problem, arising in fields ranging from machine vision to chemical engineering to neuroscience. Because GM is NP-hard, exact algorithms are unsuitable for today's massive graphs; yet, scalable GM algorithms have received short shrift.  GM can be formulated as a quadratic program with linear and binary constraints.  We develop a fast approximate quadratic (\FAQ) assignment algorithm to approximately solve a relaxed quadratic program with only linear constraints.  \FAQ scales cubicly with the number of vertices, and demonstrates marked improvements over previous state-of-the-art on nearly all benchmarks. Moreover, our non-convex formulation facilitates multiple restarts; $2-3$ wisely chosen initial conditions yield the best objective function on \emph{all} benchmarks. We find qualitatively similar results for our motivating application: brain-graph matching.  Unfortunately, the computational complexity of \FAQ scales too poorly to use it for mammalian brain-graphs, with millions or billions of vertices.  To inspire further development of approximate solutions to these problems, this work is available from on the first author's website, \url{http://jovo.me}.
\end{abstract}




\section{Introduction}

Graph matching---the process of finding an optimal permutation of the vertices of one graph to minimize adjacency disagreements with the vertices of another---is a famously computationally daunting problem (see, for example, ``Thirty Years of Graph Matching in Pattern Recognition''  \cite{Conte2004}). Specifically, graph matching is an $\mc{NP}$-hard problem, in particular, we do not know whether a polynomial time algorithm can solve it \cite{Papadimitriou1998}.  Perhaps the most prominent special case of graph matching is the traveling salesman problem \cite{Burkard2009}. As such, performance of graph matching algorithms are usually evaluated on graphs with $\dot{\approx} 10$ vertices, or at maximum $\dot{\approx} 100$ (see \cite{Burkard1997} for a description of the standard set of benchmarks).  Yet, it is increasingly popular to represent large data sets by a graph, and thus increasingly desirable to consider matching large graphs.  

The motivating application for this work is \emph{brain-graph matching}.  A brain-graph (aka, a connectome) is a graph for which vertices represent (collections of) neurons and edges represent connections between them \cite{SpornsKotter05, Hagmann05}. Via  Magnetic resonance (MR) imaging, one can image the whole brain and estimate connectivity across voxels, yielding a voxelwise connectome with up to $\dot{\approx} 10^6$ vertices and $\dot{\approx} 10^9$ edges \cite{Zuo2011}.  Comparing brains is an important step for many neurobiological inference tasks.  For example, it is becoming increasingly popular to diagnose neurological diseases via comparing brain images \cite{Csernansky2004}.  To date, however, these comparisons have largely rested on anatomical (e.g., shape) comparisons, not graph comparisons.  This is despite the widely held doctrine that many
% Yet almost immediately after the ``neuron doctrine'' was conjectured (the idea that networks of neurons comprise brains), Wernicke and others began postulating that 
psychiatric disorders are fundamentally ``connectopathies'', that is, disorders of the connections of the brain \cite{Kubicki2007,Calhoun2011,Fornito2012,Fornito2012a}. Currently available tests for connectopic explanation of psychiatric disorders  hedge upon first choosing some number of graph invariants to compare across populations. The graph invariant approach to classifying is both theoretically and practically inferior to comparing whole graphs via matching \cite{VP11_unlabeled}.  

More generally, state-of-the-art inference procedures for essentially any decision-theoretic or inference task follow from constructing interpoint dissimilarity matrices \cite{Duin2011}.  Thus, we believe that graph matching of large graphs will become a fundamental subroutine of many statistical inference pipelines operating on graphs. Because the number of vertices of these graphs is so large, exact matching is intractable.   Instead, we require inexact matching algorithms (also called ``heuristics'') that will scale polynomially or even linear \cite{Conte2004}.  We develop an approach to graph matching based on a relaxation of the quadratic programming problem (QAP).  Our approach is only cubic in the number of vertices, and outperforms previously proposed approximate graph matching heuristics on a wide range of benchmark datasets as well as our motivating application.  




\section{Graph Matching} % (fold)
\label{sec:graph_matching}


A labeled graph $G=(\mc{V},\mc{E})$ consists of a vertex set $\mc{V}$, where $|\mc{V}|=n$ is number of vertices, and an edge set $\mc{E}$. %, where $|\mc{E}| \leq n^2$. 
Note that we are not restricting our formulation to be directed or exclude self-loops. Given a pair of graphs, $G_A=(\mc{V}_A,\mc{E}_A)$ and $G_B=(\mc{V}_B,\mc{E}_B)$, where $|\mc{V}_A|=|\mc{V}_B|=n$, 
let $\Pi$ be the set of permutation functions (bijections), $\Pi=\{\pi \from \mc{V}_A \to \mc{V}_B\}$.
% $\pi: \mc{V}_1 \to \mc{V}_2$ be a permutation function (bijection), and let $\Pi$ be the set of all such permutation functions.  
Now consider the following two closely related problems:
% A pair of graphs, $G_1$ and $G_2$, are isomorphic if and only if the following \emph{isomorphism criterion} holds: there exists a $\pi \in \Pi$ such that . 
% Let $A$ be the adjacency matrix representation of graph such that $A_{ij}=1$ if there is an edge from $u$ to $v$, and $A_{ij}=0$ otherwise. 
% Note that the below follows for directed/undirected and loopy/non-loopy graphs.
% $u \sim v \in \mc{E}$ and $A_{ij}=0$ otherwise.  
% Let  $\Pi$ be the set of permutation functions, where a permutation function (bijection) $\pi: \mc{V} \to \mc{V}$ (re-)orders the elements of the set $\mc{V}$.  Given a pair of $n \times n$ adjacency matrices, $A=(a_{ij})$ and $B=(b_{ij})$, consider the following two problems:
\begin{itemize}
	\item \textbf{Graph Isomorphism (GI):}  Does there exist a $\pi \in \Pi$ such that $(u,v) \in \mc{E}_A$ if and only if $(\pi(u),\pi(v)) \in \mc{E}_B$. 
		\item \textbf{Graph Matching (GM):}
		% Which $\pi \in \Pi$ minimizes the number of pairs of vertices $u,v \in \mc{V}_A$ such that $(u,v) \in \mc{E}_A$ and $(\pi (u) ,\pi (v)) \not \in \mc{E}_B$ or $(u,v) \not \in \mc{E}_A$ and  $(\pi (u) ,\pi (v)) \in \mc{E}_B$
		 Which $\pi \in \Pi$ minimizes adjacency disagreements between $\mc{E}_A$ and the permuted $\mc{E}_B$?
\end{itemize}


Both GI and GM are computationally difficult. GM is at least as hard as GI, since solving GM also solves GI, but not vice versa. It is not known whether GI is in complexity class $\mc{P}$ \cite{Fortin1996}.  In fact, GI is one of the few problems for which, if $\mc{P} \neq \mc{NP}$, then GI might reside in an intermediate complexity class called $\mc{GI}$-complete.  GM, however, is known to be $\mc{NP}$-hard.    
 % There exist no known algorithms for which worst case behavior is polynomial \cite{Fortin1996}.  While GM is known to be $\mc{NP}$-hard, it remains unclear whether GI is in $\mc{P}$, $\mc{NP}$, or its own intermediate complexity class, $\mc{NP}$-isomorphism (or isomorphism-complete).  
Yet, for large classes of GI and GM problems, linear or polynomial time algorithms are available \cite{Babai1980}.  Moreover, at worst, it is clear that GI is only ``moderately exponential,'' for example, $\mc{O}(\exp\{n^{1/2 + o(1)}\})$ \cite{Babai1981}.  Unfortunately, even when linear or polynomial time GI or GM algorithms are available for special cases of graphs, the constants are typically unbearably large.  For example, if all vertices have degree less than $k$, there is a linear time algorithm for GI.  However, the hidden constant in this algorithm is $512k^3!$ (yes, that is a factorial!) \cite{Chen1994}.  

Because we are interested in solving GM for graphs with $\dot{\approx} 10^6$ or more vertices, exact GM solutions will be computationally intractable. As such, we develop a fast approximate graph matching algorithm.   Our approach is based on formulating GM as a quadratic assignment problem (QAP).  %Below, we introduce assignment problems, and reiterate their close relationship to GI and GM \cite{Burkard2009}.

% section graph_matching (end)


\section{Graph  Matching as a QAP} % (fold)
\label{sub:preliminaries}

% Our interests here lie in a particular instantiation of QAPs, the weighted graph matching problem \cite{Umeyama1988}.  A \emph{graph} is the mathematical abstraction of a network, consisting of a collection of vertices (or nodes)  and edges (or links, arcs) between them  \cite{Bollobas1998}.  A \emph{weighted graph},  is a kind of \emph{attributed graph}, where each edge has associated with it a weight.  The (weighted) graph matching problem (WGMP) is the problem of ``aligning'' the vertices of a pair of (weighted) graphs such that each vertex in one graph can be assigned to its corresponding vertex in the other graph.
% 
% Formally, let $G=(V,E)$ be a graph, where $V$ is the set of $|V|=n$ vertices and $E$ is the set of edges between them.  Let $i\sim j$ indicate the presence of an edge from $i$ to $j$.  The \emph{adjacency matrix} representations of a graph is a matrix, $A \in \Real^{n \times n}$, where $a_{ij}=0$ if and only if $i\sim j$.  In a weighted graph, $G=(V,E,A)$, each edge's weight can be non-binary, that is, $a_{ij} \in \Real$.

Graph matching can be formulated as a quadratic assignment problem (QAP).  Let $A=(a_{uv}) \in \{0,1\}^{n \times n}$ and $B=(b_{uv}) \in \{0,1\}^{n \times n}$ correspond to the adjacency matrix representations of two graphs that we desire to match. That is, let $a_{uv}=1$ if and only if $(u,v) \in \mc{E}_A$, and similarly for $b_{uv}$.  Moreover, let $\mc{P}$ be the set of  $n \times n$ \emph{permutation matrices}  $\mc{P}=\{P : P\T \mb{1} = P \mb{1} = \mb{1}, P \in \{0,1\}^{n \times n}\}$, where $\mb{1}$ is an $n$-dimensional column vector.
% 
% % Let , and assume that $|V(A)|=|V(B)|=n$.  
We therefore have the following problem:  
\begin{subequations} \label{eq:GM}
\begin{align}
\text{(QAP)} \quad 	&\argmin_{\pi \in \Pi} \sum_{i,j \in [n]} (a_{ij} - b_{\pi(i) \pi(j)})^2= \\
	&\argmin_{\PmcP} \norm{A - PBP\T}_F = \\
	% &\argmin_{\PmcP} \norm{PAP\T - B}_F =
	% \\&
	% \argmin_{\PmcP} \norm{PA - BP\T} =\\
	% &\argmin_{\PmcP} (PAP\T-B)\T (PAP\T-B) \\ 
	&\argmin_{\PmcP} tr(A - PBP\T)\T (A - PBP\T) = \label{eq:trQAP2} \\
	% &\argmin_{\PmcP}  tr(P\T A\T P\T P A P\T) - 2tr(PAP\T B) + tr(B\T B)  = \\ %- tr(B\T PAP\T)
	&\argmin_{\PmcP} - tr(B P\T AP), \label{eq:trQAP} %\\ % - tr(PAP\T B),			
	% &\argmin_{\PmcP} tr (A\T P\T PA) - tr(2PA) + tr(B\T B)=\\ 
	% &\argmin_{\PmcP}  - tr(B\T PAP\T)=\\
	% &\argmin_{\PmcP}  -\sum_{u \in \mc{V}} p_{ij} a_{ij} b_{ij} p_{ji} 
	% = \\ &\argmin_{\PmcP}  -\langle PAP\T, B \rangle.
	% 
	% &\argmin_{\PmcP}  -\langle B,PAP\T \rangle.
	 % =\\
\end{align}
\end{subequations}
where the last equality follows from dropping terms that cancel because $P$ is a permutation matrix. Note that the above algebraic formulation of GM facilitates generalizing the original problem statement. In particular, one can now search for the permutation that minimizes a particular objective function, $f(P)=- tr(B P\T AP)$.  Moreover, it is natural to consider ``weighted graph matching''  problems, in which each edge is associated with a weight, $a_{uv} \in \Real$.  




% Note that Eq. \eqref{eq:trQAP} demonstrates that WGMP is indeed a QAP, although the $C$ matrix has been dropped. 
Our approach follows from relaxing the above binary constraints
 % of Eq. \eqref{eq:trQAP} 
to be non-negative constraints, yielding a quadratic program with \emph{linear} constraints.  %Specifically, we relax the binary constraint to a non-negative constraint.  
Thus, the feasible region expands to the convex hull of the permutation matrices: the doubly stochastic matrices, $\mc{D}=\{P : P\T \mb{1} =  P \mb{1} = \mb{1}, P \succeq 0\}$, where $\succeq$ indicates an element-wise inequality:
% the feasible space becomes the doubly stochastic matrices; that is, all matrices whose rows and columns both sum to unity, and whose elements are all non-negative, $\mc{D}=\{P : P\T \mb{1} =  P \mb{1} = \mb{1}, P \succeq 0\}$, where $\succeq$ indicates an element-wise inequality. Because the equalities in Eq. \eqref{eq:GM} follow from $P$ being a permutation matrix, relaxing the constraints for different formulations yields different optimization problems.  We relax the binary constraints in the trace formulation, yielding:
\begin{subequations} \label{eq:FAQ}
\begin{align}
		\text{(rQAP) } \quad &\underset{P}{\text{minimize}}  && - tr(B P\T AP) \label{eq:FAQ1}  \\
		&\text{subject to } && P \in \mc{D}.
\end{align}
\end{subequations}
% FAQ---the above Fast Approximate Quadratic problem---is therefore a quadratic program with linear constraints, meaning that relatively standard solvers may be employed to search for approximately optimal solutions to an $\mc{NP}$-hard problem.
% 
% Importantly, the convex hull of permutation matrices is the set of doubly stochastic matrices, implying that this is a ``natural'' relaxation in a very meaningful sense.    Moreover, a
rQAP---the above relaxed Quadratic Assignment Problem---is quadratic but not necessarily convex, 
% Although the objective function  $f(P)= - tr(B\T PAP\T)$ of  FAQ is ,
because the Hessian of its objective function is not necessarily positive definite:
\begin{align}
	\nabla^2 f(P)  =  - B \otimes A - B\T \otimes A\T,
\end{align}
where $\otimes$ indicates the Kronecker product. This means that the solution space will potentially be multimodal, making initialization important.  With this in mind, below, we describe an algorithm to find a local optimum of rQAP.



% subsection preliminaries (end)


% We therefore determined the average complexity of our algorithm \emph{and} the leading constants.  Figure \ref{fig:scaling} suggests that our algorithm is not just cubic in time, but also has very small leading constants ($\dot{\approx} 10^{-9}$ seconds), making using this algorithm feasible for even reasonably large graphs.



\section{Fast Approximate Quadratic Assignment Problem Algorithm} % (fold)
\label{sec:faq}


Our algorithm, called \texttt{FAQ}, has three components:
\begin{enumerate}[A.]
	\item Choose a suitable initial position. % $P^{(0)} \in \mc{D}$.
	\item Find a local solution to rQAP. %, $\mh{D} \in \mc{D}$.
	\item Project onto the set of permutation matrices. %, yielding $\mh{P} \in \mc{P}$.
\end{enumerate}
% We refer to one run of the above three steps as \texttt{FAQ}.  For any integer $m$, upon using $m$ restarts, we report only the best solution, and we refer to the whole procedure as \texttt{FAQ}$_m$.  
Below, we provide details for each component.

\textbf{A: Find a suitable initial position.}  While any doubly stochastic matrix would be a feasible initial point, we choose the 
% two choices seem natural: (i) the 
``flat doubly  stochastic matrix,'' $J=\ve{1} \cdot \ve{1}\T/n$, which is the barycenter of the feasible region.
% , and (ii) the identity matrix, which is a permutation matrix.  We elect to use the barycenter as our default initial starting point.
% Therefore, if we run \FAQ  once, we always start with one of those two.  If we use multiple restarts, each initial point is ``near'' the flat matrix.  Specifically, we sample $K$, a random doubly stochastic matrix using 10 iterations of Sinkhorn balancing \cite{Sinkhorn1964}, and let $P^{(0)}=(J+K)/2$. %Given this initial estimate, we iterate the following five steps until convergence.


\textbf{B: Find a local solution to rQAP.} As mentioned above, rQAP is a quadratic problem with linear constraints.  A number of off-the-shelf algorithms are readily available for finding local optima in such problems.  We utilize the Frank-Wolfe algorithm (\texttt{FW}), a successive linear programing problem originally devised to solve quadratic problems with linear constraints \cite{Frank1956, Bradley1977}.


Although \texttt{FW} is a relatively standard solver, especially as a subroutine for QAP algorithms \cite{Anstreicher03}, below we provide a detailed view of applying \texttt{FW} to rQAP.
% , where our objective function is %. Let our objective function be that of Eq. \eqref{eq:FAQ1}, 
% $f(P)=tr(B\T PAP\T)$. 
Given an initial position, $P^{(0)}$, iterate the following four steps.

\emph{Step 1: Compute the gradient $\nabla f(P^{(i)})$:}  The gradient $f$ with respect to $P$ is given by
% \emph{Step 1: Compute the gradient} The gradient of $f$ with respect to $P$ is given by
\begin{align} \label{eq:grad}
	\nabla f (P^{(i)}) = 
	% \partial f / \partial P^{(i)} =
	  - A P^{(i)} B\T - A\T P^{(i)} B.
\end{align}


\emph{Step 2: Compute a new putative point $\mt{P}^{(i+1)}$:} The new putative point is given by the argument that minimizes a first-order Taylor series approximation to $f(P)$ around the current estimate, $P^{(i)}$. The first-order Taylor series approximation to $f(P)$ is given by
\begin{align}
	\mt{f}^{(i)}(P) \defn f(P^{(i)}) + \nabla f(P^{(i)})\T(P - P^{(i)}).
\end{align}
Thus, Step 2 of \texttt{FW} is
% \begin{align}
% 		\text{(FW1) } \quad &\underset{P}{\text{minimize}}  && \mt{f}(P)  \\
% 		&\text{subject to } && P \in \mc{D},
% \end{align}
% which is equivalent to
\begin{subequations} \label{eq:FW1}
\begin{align}
	\mt{P}^{(i+1)} &= \argmin_{P \in \mc{D}} f(P^{(i)}) + \nabla f(P^{(i)})\T(P - P^{(i)}) 
	\\ &=\argmin_{P \in \mc{D}} \nabla f(P^{(i)})\T P. \label{eq:dotFW1}
	 % \\ &=\argmin_{P \in \mc{D}}  \langle \nabla f(P^{(i)}), P \rangle, 
\end{align}
\end{subequations}
As it turns out, Eq. \eqref{eq:dotFW1} can be solved as a \emph{Linear Assignment Problem} (LAP).  The details of LAPs are well known \cite{Burkard2009}, so we relegate them to the appendix.  Suffice it to say here, LAPs can be solved via  the ``Hungarian Algorithm'', named after three Hungarian mathematicians \cite{Kuhn1955, Konig1931, Egevary1931}.  Modern variants of the Hungarian algorithm are cubic in $n$, that is, $\mc{O}(n^3)$, or even faster in the case of sparse or otherwise structured graphs \cite{Jonker1987, Burkard2009}.  The $\mc{O}(n^3)$ computational complexity of \texttt{FW} was the primary motivating factor for utilizing \texttt{FW}; generic linear programs can require up to $\mc{O}(n^7)$.

% Thus, we can solve the first step of FW upon computing 

\emph{Step 3: Compute the step size $\alpha^{(i)}$} Given $\mt{P}^{(i+1)}$, the new point is given maximizing the \emph{original} optimization problem, rQAP, along the line segment from $P^{(i)}$ to $\mt{P}^{(i+1)}$ in $\mc{D}$.    
% 
% \begin{align}
% 	d^{(i)}=L^{(i)}-P^{(i)}.
% \end{align}
% 
% % paragraph step_3_updating_the_direction (end)
% 
% \emph{Step 4: Line search} Given this direction, one can then perform a line search to find the doubly stochastic matrix that minimizes the objective function along that direction:
\begin{align}\label{eq:step}
	\alpha^{(i)} = \argmin_{\alpha \in [0,1]} f(P^{(i)} + \alpha^{(i)} \mt{P}^{(i)}).
\end{align}
This can be performed exactly, because $f$ is a quadratic function.  

% paragraph step_4_line_search (end)

\emph{Step 4: Update $P^{(i)}$} Finally, the new estimated doubly stochastic matrix is given by
\begin{align}\label{eq:update}
	P^{(i+1)} = P^{(i)} + \alpha^{(i)} \mt{P}^{(i+1)}.
\end{align}

% paragraph step_5_update_q_ (end)

\emph{Stopping criteria} Steps 1--4 are iterated until some stopping criterion is met (computational budget limits, $P^{(i)}$ stops changing much, or $\nabla f(P^{(i)})$ is close to zero).  These four steps collectively comprise the Frank-Wolfe algorithm for solving rQAP.  %Note that while $P^{(i)}$ will generally not be a permutation matrix, we do not project $P^{(i)}$ back onto the set of permutation matrices between each iteration, as that projection requires $\mc{O}(n^3)$ time.


\textbf{C: Project onto the set of permutation matrices.}   Let $\wh{D}$ be the doubly stochastic matrix resulting from the final iteration of \texttt{FW}.  We project $\wh{D}$ onto the set of permutation matrices, yielding
\begin{align} \label{eq:proj}
	\wh{P} = \argmin_{\PmcP} -\langle \wh{D}, P \rangle,
\end{align}
where $\langle \cdot,\cdot \rangle$ %the equality on the second to last line defines 
is the usual Euclidean inner product, i.e., $\langle X,Y\rangle \defn tr(X\T Y)= \sum_{ij} x_{ij} y_{ij}$.  Note that Eq. \eqref{eq:proj} is a LAP (again, see appendix for details).



\section{Results} % (fold)
\label{sec:theoretical_results}





\subsection{Algorithm Complexity and leading constants} % (fold)
\label{sub:algorithm_complexity_and_leading_constants}

% Both GM and its closely related counterpart, graph isomorphism (GI), are computationally difficult.  There exist no known algorithms for which worst case behavior is polynomial \cite{Fortin1996}.  While GM is known to be $\mc{NP}$-hard, it remains unclear whether GI is in $\mc{P}$, $\mc{NP}$, or its own intermediate complexity class, $\mc{NP}$-isomorphism (or isomorphism-complete).  Yet, for large classes of GI and GM problems, linear or polynomial time algorithms are available \cite{Babai1980}.  Moreover, at worst, it is clear that GI is only ``moderately exponential,'' for example, $\mc{O}(\exp\{n^{1/2 + o(1)}\})$ \cite{Babai1981}.  Unfortunately, even when linear or polynomial time GM or GI algorithms are available for special cases of graphs, the constants are typically unbearably large.  For example, if all graphs have degree less than $k$, there is a linear time algorithm for GI.  However, the hidden constant in this algorithm is $512k^3!$ \cite{Chen1994}.  
As mentioned above, GM is computationally difficult; even those special cases for which polynomial time algorithms are available, the leading constants are intractably large for all but the simplest cases. We therefore determined the average complexity of our algorithm and the leading constants.  Figure \ref{fig:scaling} suggests that our algorithm is not just cubic in time, but also has very small leading constants ($\dot{\approx} 10^{-9}$ seconds), making using this algorithm feasible for even reasonably large graphs.



\begin{figure}[htbp]
	\centering			
	\includegraphics[width=0.5\linewidth]{../figs/ErdosRenyi_results.pdf}
	\caption{Running time of \FAQ as function of number of vertices. Data was sampled from an Erd\"os-R\'enyi model with $p=log(n)/n$.  Each dot represents a single simulation, with 100 simulations per $n$.  The solid line is the best fit cubic function.  Note the leading constant is $\dot{\approx} 10^{-9}$ seconds. \FAQ finds the optimal objective function value in every simulation.}
	\label{fig:scaling}
\end{figure}

% subsection algorithm_complexity_and_leading_constants (end)


\subsection{QAP Undirected Benchmarks}
\label{sub:undirected}

We next assess the computational properties of \FAQ in comparison with other the previous state-of-the-art algorithms.  We therefore compare \FAQ to other approaches using a selection of the QAP benchmark library, QAPLIB \cite{Burkard1997}.  Specifically, \cite{Zaslavskiy2009} created a path following algorithm (\texttt{PATH}) based on a convex and concave relaxation of QAP.  In that manuscript, the authors considered 16 datasets from the QAPLIB benchmark, the same 16 datasets as were used in \cite{Schellewald2001}, which are known to be ``particularly difficult''.  \texttt{PATH} was shown to outperform other state-of-the-art algorithms on 14 of 16 tests.  Specifically, \texttt{PATH} was compared to the Quadratic Programming Bound approach (\texttt{QGP}) of \cite{Anstreicher2001}, the graduated assignment algorithm (\texttt{GRAD}) of \cite{Gold1996}, and Umeyama's algorithm (\texttt{U}) \cite{Umeyama1988}.  Because either \texttt{PATH} or \texttt{QBP} outperformed \texttt{GRAD} and \texttt{U} on every dataset, Table \ref{tab:1} shows the performance of \FAQ versus \texttt{PATH} and \texttt{QBP}.  \FAQ outperforms both of the previous state-of-the-art cubic algorithms on 13 out of 16 benchmarks.  Figure \ref{fig:path16} presents the same data graphically. The top panel compares both \FAQ and \texttt{PSOA}---which is the minimum of the previous state-of-the-art (either \texttt{PATH} or \texttt{QBP} here)---to the absolute minimum; \FAQ get closer than \texttt{PSOA} to the minimum on 13 of 16. The bottom panel shows the ratio of \FAQ to \texttt{PSOA}. 


\begin{table}[h!]
\caption{Comparison of \FAQ with the optimal objective function value and previous state-of-the-art on a set of 16 standard benchmarks from QAPLIB.  The best (lowest) value is in \textbf{bold}. The number of vertices for each problem is the number in its name (second column).}
\begin{center}
\begin{tabular}{|r|r|r||l|l|l|l|l|}
\hline
\# & Problem  &   Optimal   & \FAQ & \texttt{PATH} & \texttt{QBP} \\
\hline
1&    chr12c &   11156 &    \textbf{13072} &   18048 	& 20306\\
2&    chr15a &    9896 &    27584 &   \textbf{19086} 	& 26132\\
3&    chr15c &    9504 &    \textbf{11936}  &   16206 	& 29862\\
4&   chr20b &    2298 & \textbf{3068} &    5560 		& 6674\\
5&    chr22b &    6194 &    \textbf{8482} &    8500 		& 9942\\
6&    esc16b & 292 &    320 & 300 		& \textbf{296}\\
7& rou12 &  235528 &    \textbf{253684} &  256320 	& 278834\\
8& rou15 &  354210 &    \textbf{371458} &  391270 	& 381016\\
9& rou20 &  725522 &    \textbf{743884} &  778284 	& 804676\\
10&    tai10a &  135028 &   157954 &  \textbf{152534} 	& 165364\\
11&    tai15a &  388214 &   \textbf{397376} &  419224 	& 455778\\
12&    tai17a &  491812 &   \textbf{529134} &  530978 	& 550862\\
13&    tai20a &  703482 &   \textbf{734276} &  753712 	& 799790\\
14&    tai30a & 1818146 &  	\textbf{1894640} & 1903872 	& 1996442\\
15&    tai35a & 2422002 & 	\textbf{2460940} & 2555110 	& 2720986\\
16&    tai40a & 3139370 &  	\textbf{3227612} & 3281830 	& 3529402\\
    \hline
\end{tabular}
\end{center}
\label{tab:1}
\end{table}%


\begin{figure}[htbp]
	\centering
		\includegraphics[width=0.5\linewidth]{../figs/path16.pdf}
	\caption{Performance of \FAQ relative to the previous state-of-the-art (\texttt{PSOA}) algorithms on the undirected QABLIB benchmarks.  Top: The optimal error ratio is defined: $(\mh{f} - f^*)/f^*$, for $\mh{f}$ being the minimum function value found by each algorithm, and $f^*$ is the optimal (minimum) value.  Bottom: Ratio of \FAQ minimum to \texttt{PSOA} minimum.  Note that both panels indicate that \FAQ gets closer to the minimum on 13 of 16 benchmarks.}
	\label{fig:path16}
\end{figure}


\subsection{QAP Directed Benchmarks}
\label{sub:directed}

Nothing in the development of our algorithm depends on the graphs being simple; indeed, \FAQ applies equally well to directed graphs.  To assess the performance of \FAQ on directed graphs, we compare the performance of our algorithm to the previous state-of-the-art. Liu et al.  recently developed an extended path following algorithm for directed graphs \cite{Liu2012}. They compare the performance of their algorithm (\texttt{EPATH}) with several other algorithms on a set of 16 benchmarks from QAPLIB.  In particular, they consider \texttt{U} and \texttt{GRAD}, as well as an algorithm called \texttt{QCV}, which solves a convex relaxation similar to our approach.  The \texttt{EPATH} algorithm achieves at least as low objective value as the other algorithms on 15 of 16 benchmarks.  Our algorithm, \texttt{FAQ}, always gets the best of the five algorithms.  Table \ref{tab:directed} shows the numerical results comparing \FAQ to \texttt{EPATH} and \texttt{GRAD}, which sometimes did better than \texttt{EPATH}.  Note that some of the algorithms achieve the absolute minimum on some benchmarks.  Figure \ref{fig:lipa16} compares \FAQ to whichever other algorithm did best, clearly indicating that \FAQ is the best on these benchmarks.


\begin{table}[h!]
\caption{Comparison of \FAQ with optimal objective function value and previous state-of-the-art for undirected graphs.  The best (lowest) value is in \textbf{bold}. Asterisks indicate achievement of the global minimum.  The number of vertices for each problem is the number in its name (second column).}
\begin{center}
\begin{tabular}{|r|r|r||l|l|l|l|l|}
	\hline 
	          \# &  Problem &      Optimal & \texttt{FAQ} & \texttt{EPATH} & \texttt{GRAD} \\
	\hline 
	           1 &  lipa20a &     3683 & \textbf{3791} &     3885 &     3909 \\ 
	           2 &  lipa20b &    27076 & \textbf{27076}$^*$ &    32081 &    \textbf{27076}$^*$ \\ 
	           3 &  lipa30a &    13178 & \textbf{13571} 	&    13577 &    13668 \\ 
	           4 &  lipa30b &   151426 & \textbf{151426}$^*$ & \textbf{151426}$^*$ &   \textbf{151426}$^*$ \\ 
	           5 &  lipa40a &    31538 & \textbf{32109} 	&    32247 &    32590 \\ 
	           6 &  lipa40b &   476581 & \textbf{476581}$^*$ &   \textbf{476581}$^*$ &   \textbf{476581}$^*$ \\ 
	           7 &  lipa50a &    62093 & \textbf{62962} &    63339 &    63730 \\ 
	           8 &  lipa50b &  1210244 & \textbf{1210244}$^*$ &  \textbf{1210244}$^*$ &  \textbf{1210244}$^*$ \\ 
	           9 &  lipa60a &   107218 & \textbf{108488} &   109168 &   109809 \\ 
	          10 &  lipa60b &  2520135 & \textbf{2520135}$^*$ &  \textbf{2520135}$^*$ &  \textbf{2520135}$^*$ \\ 
	          11 &  lipa70a &   169755 & \textbf{171820} &   172200 &   173172 \\ 
	          12 &  lipa70b &  4603200 & \textbf{4603200}$^*$ &  \textbf{4603200}$^*$ &  \textbf{4603200}$^*$ \\ 
	          13 &  lipa80a &   253195 & \textbf{256073} &   256601 &   258218 \\ 
	          14 &  lipa80b &  7763962 & \textbf{7763962}$^*$ &  \textbf{7763962}$^*$ &  \textbf{7763962}$^*$ \\ 
	          15 &  lipa90a &   360630 & \textbf{363937} &   365233 &   366743 \\ 
	          16 &  lipa90b & 12490441 & \textbf{12490441}$^*$ & \textbf{12490441}$^*$ & \textbf{12490441}$^*$ \\ 
	\hline
	\end{tabular}
\end{center}
\label{tab:directed}
\end{table}%


\begin{figure}[htbp]
	\centering
		\includegraphics[width=0.5\linewidth]{../figs/lipa16.pdf}
	\caption{Performance of \FAQ relative to the previous state-of-the-art (\texttt{PSOA}) algorithms on the undirected QABLIB benchmarks. Top and Bottom panels as in Figure \ref{fig:path16}.  Note that \FAQ gets closer to the minimum on all 8 benchmarks for which the \FAQ and \texttt{PSOA} answer differ.}
	\label{fig:lipa16}
\end{figure}



\subsection{rQAP solves QAP in certain special cases} % (fold)
\label{sub:rqap_solves_qap_}

The above numerical results can be strengthened by the below theoretical results.  
Note that rQAP relaxes the constraints of Eq. \eqref{eq:trQAP}, which suggests that in certain important special cases, the minimum of rQAP will be identical to the minimum of QAP. On the other hand, the equality between Eq. \eqref{eq:trQAP2} and Eq. \eqref{eq:trQAP} follows from dropping cross-terms that fall out of the optimization because $P$ is constrained to be a permutation matrix.  If we had relaxed the constraints prior to canceling those terms, this equality would not follow.  This leads us to wonder in which circumstances are the objective functions of QAP and rQAP equal.  The following lemma clarifies:
 % This insight leads to the following theorem:
% Although, rLAP and LAP are always equivalent, in general, it is not the case that FAQ and QAP are equivalent.  However, in a certain important special case, FAQ and QAP are equivalent.
\begin{lem}
	If $A$ and $B$ are the adjacency matrices of simple graphs (symmetric, hollow, and binary) that are isomorphic to one another, then the minimum of rQAP is equal to the minimum of QAP.
\end{lem}
\begin{proof}
Because any feasible solution to QAP is also a feasible solution to rQAP, we must only show that the optimal objective function value to rQAP can be no better than the optimal objective function value of QAP.  Let $A=PBP\T$, so that $\langle A, PBP\T\rangle=2m$, where $m$ is the number of edges in $A$.  If rQAP could achieve a lower objective value, then it must be that there exists a $D \in \mc{D}$ such that $\langle A, DBD\T\rangle > \langle A, PBP\T\rangle = 2m$ (remember that we are minimizing the negative Euclidean inner product). For that to be the case, it must be that $(DBD\T)_{ij} \geq 1$ for some $(u,v)$.  That this is not so may be seen by the submultiplicativity of the norm induced by the $\ell_{\infty}$ norm:
$\norm{Dx}_\infty \leq \norm{D}_{\infty,\infty} \norm{x}_\infty$.  Applying this twice (once for each doubly stochastic matrix multiplication) yields our result.
% Consider $d_i=\langle D, \text{col}_i(BD\T) \rangle$, where $\text{col}_i(\cdot)$ indicates the $i^{th}$ column of the matrix.  $d_i \leq 1$ for all $i \in [n]$, therefore, our result holds.
\end{proof}
% subsection rqap_solves_qap_ (end)


% section theoretical_results (end)

% \section{Numerical Results} % (fold)
% \label{sub:numerical_results}


% subsection numerical_results (end)


\subsection{Multiple Restarts} % (fold)
\label{sub:multiple_restarts}

Although \FAQ outperformed \texttt{PSOA} on 13 of 16 undirected benchmarks, and always did the best amongst 16 of 16 directed benchmarks, it was annoying to us that we did not do best on all 32 benchmarks.  
% Note that the computational bottleneck of both \FAQ and \texttt{PATH} is the Hungarian algorithm which solves a LAP. 
% \FAQ strives to solve a non-convex problem.
% In \texttt{PATH}, the algorithm finds the minimum of a convex path between two extremes, $F_0$ and $F_1$.  Similarly, \texttt{QBP} finds the minimum of a convex program.  Our approach, on the other hand, does not construct a convex problem to solve, rather, it chooses an initial starting point and then finds a local optimum (note that the initial position of the \texttt{PATH} algorithm could also be variable, because $F_0$ is not convex as they assert, so their starting point depends on their initialization). 
% 
We utilize the non-convexity of rQAP is as a feature, although it can equally well be regarded as a bug  (because rQAP is non-convex so the solution found by \FAQ depends on the initial condition).  It is a feature, however, if (i) we have some reason to believe that better solutions exist (many algorithms efficiently compute relatively tight lower bounds \cite{Anstreicher2009}), and (ii) we can efficiently search the space of initial conditions.  Although we  lack any supporting theory of optimality, we do know how to sample feasible starting points.  Specifically, we desire that our starting points are ``near'' the flat matrix, and satisfy the conditions.  Therefore, we  sample $K \in \mc{D}$, a random doubly stochastic matrix using 10 iterations of Sinkhorn balancing \cite{Sinkhorn1964}, and let our initial guess be $P^{(0)}=(J+K)/2$, where $J$ is the doubly flat matrix.  We can therefore use any number of restarts with this approach.  

Table \ref{tab:2} shows the performance of running \FAQ 3 and 100 times, reporting only the best result (indicated by \texttt{FAQ}$_3$ and \texttt{FAQ}$_{100}$, respectively), and comparing it to the best performing result from Table \ref{tab:1}.  It only required three restarts to outperform all other cubic algorithms on all 16 of 16 benchmarks.  Moreover, after 100 restarts, \FAQ finds the absolute minimum on 3 of the 16 benchmarks. Figure \ref{fig:restarts} graphically demonstrates these results. 
 Note that restarting \FAQ a fixed number of multiple times is still cubic.  Future work could investigate performance as a function of the number of restarts. %, although with an arbitrary number of random restarts, stating that it is cubic is somewhat meaningless.  



\begin{table}[h!]
\caption{Comparison of \FAQ with optimal objective function value and the best result from Table \ref{tab:1} on undirected benchmarks.  Note that \FAQ restarted 100 times finds the optimal objective function value in 3 of 16 benchmarks, and that \FAQ restarted 3 times finds a minimum better than the PSOA on all 16 benchmarks.}
\begin{center}
\begin{tabular}{|r|r|r||l|l|l|l|l|}
\hline
\# & Problem  &   Optimal    & \texttt{FAQ}$_{100}$ & \texttt{FAQ}$_{3}$ & min(\FAQ,\texttt{PSOA}) \\
\hline
1&    chr12c &   11156 &    \textbf{12176} &   13072 & 13072 \\
2&    chr15a &    9896 &    \textbf{9896}$^*$ &   17272 &  19086 \\
3&    chr15c &    9504 &    \textbf{10960} &   14274 &  16206 \\
4&   chr20b &    2298 &     \textbf{2786} &    3068 &    3068 \\
5&    chr22b &    6194 &    \textbf{7218} &    7876 &   8482 \\
6&    esc16b & 	292 & 		\textbf{292}$^*$ & 294 &    296 \\
7& 	   rou12 &  235528 &  \textbf{235528}$^*$ &  238134 &    253684 \\
8& 	   rou15 &  354210 &  \textbf{356654} &  371458 &    371458 \\
9&      rou20 &  725522 &  \textbf{730614} &  743884 &    743884 \\
10&    tai10a &  135028 &  \textbf{135828} &  148970 &    152534 \\
11&    tai15a &  388214 &  \textbf{391522} &  397376 &    397376 \\
12&    tai17a &  491812 &  \textbf{496598} &  511574 &    529134 \\
13&    tai20a &  703482 &  \textbf{711840} &  721540 &    734276 \\
14&    tai30a & 1818146 & \textbf{1844636} & 1890738 &  1894640 \\
15&    tai35a & 2422002 & \textbf{2454292} & 2460940 &  2460940 \\
16&    tai40a & 3139370 & \textbf{3187738} & 3194826 &  3227612 \\
    \hline
\end{tabular}
\end{center}
\label{tab:2}
\end{table}%

\begin{figure}[htbp]
	\centering
		\includegraphics[width=0.5\linewidth]{../figs/path16_restarts.pdf}
	\caption{Performance of \FAQ with multiple restarts on the undirected benchmarks. \texttt{FAQ}$_3$ yields a lower objective function value than the best result from Figure \ref{fig:path16}, and \texttt{FAQ}$_{100}$ finds the absolute optimal permutation on 3 of the 16 benchmarks.  Note that no other algorithm compared ever found the optimal for any of the benchmarks.}
	\label{fig:restarts}
\end{figure}


% subsection multiple_restarts (end)

% 
% \begin{figure}[htbp]
% 	\centering			
% 	\includegraphics[width=1.0\linewidth]{../figs/benchmarks.pdf}
% 	\caption{\texttt{FAQ}$_3$ outperforms the previous state-of-the-art (PSOA) on all 16 benchmark graph matching problems.  Moreover, \FAQa outperforms PSOA on 12 of 16 tests.  For 3 of 16 tests, \FAQb achieves the minimum (none of the other algorithms ever find the absolute minimum), as indicated by a black dot.  Let $f_*$ be the minimum and $\mh{f}_x$ be the minimum achieved by algorithm $x$.  Error is $\mh{f}_x/f_*-1$.  }
% 	\label{fig:fwpath}
% \end{figure}



\subsection{Brain-Graph Matching} % (fold)
\label{sub:connectome_classification}

A ``connectome'' is a brain-graph in which vertices correspond to (collections of) neurons, and edges correspond to connections between them. The \emph{Caenorhabditis elegans} (\emph{C. elegans}) is a small worm (nematode) with $302$ labeled vertices.  We consider the subgraph with $279$ somatic neurons that form edges with other neurons \cite{WhiteBrenner86, Varshney2011}.  Two distinct kinds of edges exist between vertices: chemical and electrical ``synapses'' (edges). Any pair of vertices may have several edges of each type. Moreover, some of the synapses are hyper-edges amongst more than two vertices.   Thus, the connectome of a \emph{C. elegans} may be thought of as a weighted multi-hypergraph, where the weights are the number of edges of each type.  \FAQ natively operates on weighted or unweighted graphs.  We therefore conducted the following synthetic experiments.  

Let $A_{ij;z} \in \{0,1,2,\ldots\}$ be the number of synapses from neuron $i$ to neuron $j$ of type $z$ (either chemical $c$ or electrical $e$), and let $A_z=\{A_{ij;z}\}_{i,j \in [279]}$ for $z \in \{e,c\}$ correspond to the electrical or chemical connectome.  To generate synthetic data, we let $B_z^{(k)}=Q_z^{(k)} A_z {Q_z^{(k)}}\T$, for some $Q_z^{(k)}$ chosen uniformly at random from $\mc{P}$, effectively shuffling the vertex labels of the connectome.  Then, we try to graph match $A_z$ to $B_z^{(k)}$, for $z \in \{e,c\}$ and for $k =1,2,\ldots, 1000$, that is, we repeat the experiment $1000$ times.  We define accuracy as the fraction of vertices correctly assigned. We always start with the doubly flat matrix.
% by the value of our objective function, $f(P_z^{(k)})$.  
% To evaluate the impact of multiple restarts, for both connectomes, we restarted \FAQ up to 30 times.   Specifically, our stopping criteria on the number of restarts was either (i) perfect assignment or (ii) 30 restarts.

% Table \ref{tab:1} shows the mean (standard deviation) of accuracy and solution time for both connectomes.  For the chemical connectome, \FAQ always found the optimal solution, but not so for the electrical connectome.  


\begin{figure}[htbp]
	\centering
		\includegraphics[width=1.0\linewidth]{../figs/connectomes.pdf}
	\caption{Performance of \texttt{U}, \texttt{QCV}, \texttt{PATH}, and \FAQ on synthetic C.~elegans connectome data, that is, graph matching the true connectomes with permuted versions of themselves.  Error is the fraction of vertices correctly matched.  Circle indicates the median, thick black bars indicate the quartiles, thin black lines indicate extreme but non-outlier points, and plus signs are outliers. The top panels indicate error (fraction of misassigned vertices), and the bottom panels indicate wall time on a 2.2 GHz Apple MacBook.  The left panels show the chemical connectome results, and the right panels show the electrical connectome results. For the chemical connectome, \FAQ always obtained the optimal solution, whereas none of the other algorithms ever found the optimal.  On the other hand, for the electrical connectome, none of the algorithms ever found the optimal.  \FAQ also ran very quickly, nearly as quickly as \texttt{U} and \texttt{QCV}, and much faster than \texttt{PATH}, even though our \FAQ implementation is in Matlab, and the others are in C.}
	\label{fig:connectomes}
\end{figure}


Figure \ref{fig:connectomes} displays the results of \FAQ along with three previous state-of-the-art algorithms on the two connectomes: (i) Umeyama's algorithm\texttt{U}, (ii) a quadratic convex relaxation \texttt{QCV} (which follows from relaxing the permutation matrix constraint to the doubly stochastic constraint in Eq. \eqref{eq:trQAP2}), and (iii) \texttt{PATH}.  The top left panel indicates that \FAQ \emph{always} found the optimal solution for the chemical connectome, whereas none of the other algorithms \emph{ever} found the optimal solution.  On the other hand, the top right panel shows that for the electrical connectome, none of the four algorithms ever found the optimal. One hundred restarts of \FAQ failed to significantly improve the results. 

The bottom panels compare the wall time of the various algorithms, running on an 2.2 GHz Apple MacBook. Note that we have only a Matlab implementation of \texttt{FAQ}, whereas the other algorithms are implemented in C.  Nonetheless, \FAQ runs nearly as quickly as both \texttt{U} and \texttt{QCV}, and significantly faster than \texttt{PATH}, for both connectomes.  %This suggests that lower level language implementation of \FAQ might be 

 % \FAQ ran very quickly, finding the optimal solution for the chemical connectome and converging for the electrical connectome, in only a few seconds.  Although computer times are not directly comparable, as our \FAQ implementation is in Matlab, and the other algorithms are coded in C, consider \FAQ versus \texttt{PATH}.  The bottleneck in both is \texttt{FW}.  In \texttt{PATH}, the number of \texttt{FW}s depends on a parameter, $d\lambda$, which sets the step size along the path. The \texttt{PATH} paper, \cite{Zaslavskiy2009}, describes an adaptive scheme for updating $d\lambda$, with a lower bound of $10^{-5}$, clearly indicating that sometimes, \texttt{FW} is run many times.  On the other hand, in \texttt{FAQ}, \texttt{FW} is always run only once.  

The properties of these connectomes are analyzed in \cite{Varshney2011}; a cursory evaluation of the properties of these graphs does not suggest to us why the chemical connectome was so much easier to graph match than the electrical one. 


To investigate the performance of \FAQ on undirected graphs, we ran \FAQ on binarized symmeterized versions of the graphs ($A_{ij;z}=1$ if and only if $A_{ij;z}\geq 1$ or $A_{ji;z} \geq 1$).  The resulting errors are nearly identical to those presented in Figure \ref{fig:connectomes}, although speed increased by greater than a factor of two. Note that the number of vertices in this brain-graph matching problem---279---is several times larger than the largest of the 32 benchmarks used above. 




\section{Discussion}
\label{sec:discussion}

This work presents a fast approximate quadratic assignment problem algorithm called \FAQ for approximately solving large graph matching problems, motivated by brain-graphs.  Our key insight was to relax the binary constraint of QAP to its continuous and non-negative counterpart---the doubly stochastic matrix---which is the convex hull of the original feasible region.  
Numerically, we demonstrated that not only is \FAQ cubic in time, but also its leading constants are quite small---$10^{-9}$---suggesting that it can be used for graphs with hundreds or thousands of vertices.  Moreover, it achieves better performance than previous state-of-the-art cubic-time algorithms on 29 of the 32 standard QAP benchmarks, including both directed and undirected graph matching problems.  Because rQAP is non-convex, we also consider multiple restarts, and achieve improved performance for the remaining three benchmarks using only two or three restarts.  We then demonstrate that the solution to our relaxed optimization problem, rQAP, is identical to that for QAP whenever the two graphs are simple and isomorphic to one another.  Finally, we used it to match C.~elegans connectomes to permuted versions of themselves. For the chemical connectome, of the four state-of-the-art algorithms considered, \FAQ achieved perfect performance $100\%$ of the time, whereas none of the other three algorithms ever achieved perfect performance. On the other hand, all the algorithms struggled with the electrical connectome.   Moreover, \FAQ ran about as fast as two of them, and significantly faster than \texttt{PATH}, even though \FAQ is implemented in Matlab, and the others are implemented in C.  Note that these connectomes have 302 vertices, several-fold more than even the largest benchmarks. 

% our motivating application: brain-graph matching.  \FAQ solved a brain-graph matching problem, which has an order of magnitude more vertices than any of the 16 QAP benchmarks.

% These insights led to an approximate QAP solver with a few distinct features. %While others have incorporated the FW algorithm as a subroutine of a graph matching strategy \cite{Zaslavskiy2009}, we modified the FW algorithm for GM in a few ways.  

% First, after finding a local solution to the relaxed problem, we project the resulting doubly stochastic matrix onto the set of permutation matrices.  Second, we initialize the algorithm using either the identity matrix or the doubly flat matrix (the matrix where all elements are $1/n$).  These choices seem to us to be the most obvious places to start if one must choose.  Third, if one of those choices does not work, we restart FW with other ``nearby'' initial points.  These modifications facilitate improved performance on \emph{all} the benchmarks we considered.  Moreover, although the algorithm scales cubically with the number of vertices, the leading constants are very small ($\mc{O}(10^{-9})$ seconds), so the algorithm runs quite fast on reasonably sized networks (e.g., $n \dot{\approx} 100$).  Indeed, on a biologically inspired GM problem, \emph{C. elegans} connectome mapping, this approach was both fast and effective.  

Fortunately, our work is not done. Even with very small leading constants for this algorithm, as $n$ increases, the computational burden gets quite high.  For example, extrapolating the curve of Figure \ref{fig:scaling}, this algorithm would take about 20 years to finish (on a standard laptop from 2011) when $n=100,000$.  We hope to be able to approximately solve rQAP on graphs much larger than that, given that the number of neurons even in a fly brain, for example, is $\approx 250,000$.  More efficient algorithms and/or implementations are required for such massive graph matching. 

% more efficient algorithms and/or implementations are essential.

% Although \FAQm consistently found the optimal solution for the \emph{C. elegans} chemical connectome, connectomes for different organisms even within a species are unlikely to be identical. Even if all connectomes of a particular species were identical, measurement error will likely persist \cite{Helmstaedter2011}. Therefore, \texttt{FAQ}$_m$'s scientific utility will largely rest on its performance under noisy conditions, which we aim to explore in future work.  

Additional future work might generalize \FAQ in a number of ways.  First, many (brain-) graphs of interest will be errorfully observed \cite{Priebe2011}, that is, vertices might be missing and putative edges might exhibit both false positives and false negatives.  Explicitly dealing with this error source is both theoretically and practically of interest \cite{VP11_unlabeled}.  
% that QAP and LAP are so similar suggests that perhaps one could simply implement a single iteration of QAP starting from the identity.  While not changing the order of complexity, it could reduce computational time by at least an order of magnitude, without drastically changing performance properties (because convergence typically requires around $5-15$ iterations for the graphs we tested).  The relative performance/computational cost trade-off merits further theoretical investigations.  
Second, for many brain-graph matching problems, the number of vertices will not be the same across the brains.  Recent work from \cite{Zaslavskiy2009, Zaslavskiy2010} and \cite{Escolano2011} suggest that extensions in this direction would be both relatively straightforward and effective. Third, the most ``costly'' subroutine is LAP.  Fortunately, LAP is a quadratic optimization problem with linear constraints.  A number of parallelized optimization strategies could therefore potentially be brought to bear on this problem \cite{Boyd2011}.  Fourth, our matrices have certain special properties, namely sparsity, which makes more efficient algorithms (such as ``active set'' algorithms) readily available for further speed increases.  Fourth, for brain-graphs, we have some prior information that could easily be incorporated in the form of vertex attributes.  For example, position in the brain, cell type, etc., could be used to measure ``dissimilarity'' between vertices.  %The WGMP could easily incorporate these dissimilarities, in fact, the original QAP formulation already encodes them via the matrix $C$; that matrix was simply dropped when WGMP was originally proposed.  
% The objective function could then be modified to give
% \begin{align} \label{eq:Jqap}
% 	\mt{Q}_{AB}= \argmin_{Q \in \mc{D}} \norm{Q A Q\T - B}^2_F + \lambda J(Q),
% \end{align}
% where $J(Q)$ is a dissimilarity based penalty and $\lambda$ is a hyper-parameter.  
Finally, although this approach natively operates on both unweighted and weighted graphs, multi-graphs are a possible extension.

In conclusion, this manuscript has presented an algorithm for approximately solving the quadratic assignment problem that is fast, effective, and easily generalizable.  Yet, the $\mc{O}(n^3)$ complexity remains too slow to solve many problems of interest.  To facilitate further development and applications, all the code and data used in this manuscript is available from the first author's website, \url{http://jovo.me}.

% \subsection{Related Problems} % (fold)
% \label{sub:related_problems}
% 
% In addition to brain-graph matching, large approximate graph matching could be fruitful for a number of other domains.  For example, consider social networks.  In both the Twitter and Facebook graph, each vertex represents an individual.  For many users of each social networking application, it is not clear whether they have an account on another social network, and if so, what is the label of that account.  Thus, graph matching in this domain could suggest assignments of Facebook user names to twitter accounts.  Alternately, consider a language graph, where each vertex represent a word in some language, and an edge represent the existence of an adjacency between a pair of words in a text corpus of that language.  If one could match a pair of graphs corresponding to two different languages, one might obtain a highly effective machine translation tool. This might be especially true when no additional information is known about one or both languages.
% 
% Consider the scale of these problems.  Twitter and Facebook have $\mc{O}(10^8)$ users and languages often have $\mc{O}(10^5)$ words.  Exact graph matching algorithms require exponential time in the worst case.  So, even considering the smallest problem above, brain-graph matching, the fastest exact graph matching algorithms would require more time than there are nanoseconds since the big bang.\footnote{Assuming the big bang occurred 15 billion years ago means about $10^{17}$ seconds ago.  Assuming computational time is $1.004^n$ nanoseconds, even when $n=10^4$, the problem will already exceed the number of seconds since the big bang} This motivates us to consider developing \emph{approximate} (or \emph{heuristic}) algorithms, with polynomial time complexity.  Indeed, we present here an algorithm the requires approximately one week on a standard desktop computer (running non-optimized code) to match graphs with $\mc{O}(10^4)$ vertices.
% 
% 
% % subsection related_problems (end)


\appendix

% \textbf{APPENDIX}
\section{Linear Assignment Problems} % (fold)
% \label{ssub:linear_assignment_problems}

% subsubsection linear_assignment_problems (end)

The standard way of writing a Linear Assignment Problem (LAP) is
\begin{subequations} \label{eq:LAP}
\begin{align}
	 \text{(LAP) }\quad  &\underset{\pi}{\text{minimize}} \sum_{u,v \in [n]} a_{u \pi(v)} b_{ij} \\
	&\text{subject to } \pi \in \Pi,
\end{align}
\end{subequations}
which can be written equivalently in a number of ways using the notion of permutation matrix introduced in the main text:
\begin{subequations} \label{eq:LAP2}
\begin{align}
	&\argmin_{\PmcP} \norm{PA - B}_F =\\
	&\argmin_{\PmcP} \, tr(PA-B)\T (PA-B)=\\ 
	% &\argmin_{\PmcP} tr (A\T P\T PA) - tr(2PAB\T) + tr(B\T B)=\\ 
	&\argmin_{\PmcP}  -tr (P AB\T) = \argmin_{\PmcP}  -\langle P\T, AB\T \rangle = \label{eq:2c} \\
	% &\argmin_{\PmcP}  -\sum_{u,v \in [n]} p_{ij} a_{ij} b_{ji}
	% =\\% &\argmin_{\PmcP}  - \text{vec}(P)\T \text{vec}(AB\T).=\\
	&\argmin_{\PmcP}  -\langle P, AB\T \rangle, \label{eq:dotLAP}
\end{align}
\end{subequations}
where $\langle \cdot,\cdot \rangle$ %the equality on the second to last line defines 
is the usual Euclidean inner product, i.e., $\langle X,Y\rangle \defn tr(X\T Y)= \sum_{ij} x_{ij} y_{ij}$.
While the objective function and the first two constraints of LAP are linear, the binary constraints make solving even this problem computationally tricky.  Nonetheless, in the last several decades, there has been much progress in accelerating algorithms for solving LAPs, starting with exponential time, all the way down to $\mc{O}(n^3)$ for general LAPs, and even faster for certain special cases (e.g., sparse matrices) \cite{Jonker1987, Burkard2009}.

That Eq. \eqref{eq:dotFW1} is a LAP is evident by considering Eq. \eqref{eq:dotLAP}.  If $A=\nabla_P^{(i)}$ and $B=I$ (the $n\times n$ identity matrix), then Eq. \eqref{eq:dotFW1} is identical to Eq. \eqref{eq:dotLAP}.


% The last form indicates that LAP is a linear programming problem (hence the name).  Yet, the constraints, $\mc{P}$, make it a bit trickier.  The feasible region $\mc{P}$ can be written as a set of three constraints: two linear equality constraint sets and a binary constraint.  The LAP objection function with constraints can explicitly be written:
% \begin{align}
% 		&\text{minimize}_P  &&\sum_{u \in \mc{V}} -p_{ij} a_{ij} b_{ji} \nonumber \\
% 		&\text{subject to } && \sum_{u \in \mc{V}} p_{ij} = 1 \, \forall u \in \mc{V} \nonumber \\
% 		& && \sum_{v \in \mc{V}} p_{ij} = 1 \, \forall v \in \mc{V}, \nonumber \\
% 		& &&p_{ij} \in \{0,1\} \, \forall u,v. \label{eq:rLAP}	
% \end{align}
% Perhaps because LAP comes up in a wide variety of contexts, a large number of algorithms have been developed to solve LAP \cite{Burkard2009}.  These algorithms have become increasing efficient.  
% One of the most popular algorithms, the so-called ``Hungarian algorithm'' has time complexity $\mc{O}(n^3)$ \cite{Jonker1987}.  Under certain conditions (for example, when $AB\T$ is sparse), faster implementations are also available.  As will be seen below, LAP is a key subroutine to our inexact QAP solution.  

To solve a LAP, consider a continuous relaxation of LAP, specifically, relaxing the permutation matrix constraint to a doubly stochastic matrix constraint:
% A matrix $P$ is doubly stochastic precisely when $P$ satisfies the following three conditions: 
% \begin{enumerate}
% \item	$P\mb{1} = \mb{1}$,
% \item	$P\T \mb{1}=\mb{1}$, %\\
% \item 	$P \in  \Real_+^{n \times n}$,
% \end{enumerate}
% where the third constraint relaxes the binary constraints of the permutation matrices with a non-negativity constraint.  
% Let $\mc{D}$ be the set of doubly stochastic matrices.
% With this, we now state a relaxed LAP problem:
\begin{subequations} \label{eq:rLAP}
\begin{align}
		\text{(rLAP) } \quad &\underset{P}{\text{minimize}}  &&-\langle P, AB\T \rangle \\
		&\text{subject to } && P \in \mc{D}.
		% && \sum_{u \in \mc{V}} p_{ij} = 1 \, \forall u \in \mc{V} \nonumber \\
		% 		& && \sum_{v \in \mc{V}} p_{ij} = 1 \, \forall v \in \mc{V}, \nonumber \\
		% 		& &&p_{ij} \geq 0 \, \forall u,v, \label{eq:ALAP}	
\end{align}
\end{subequations}
As it turns out, solving rLAP is equivalent to solving LAP.
\begin{prop}
	LAP and rLAP are equivalent, meaning that they have the same optimal objective function value.
\end{prop}
\begin{proof}
	Although this proposition is typically proven by invoking total unimodularity, we present a simpler proof here.	Let $P'$ be a solution to LAP and let $P = \sum_{i\in[k]} \alpha_i P^{(i)}$ be a solution to rLAP for some positive integer $k$, permutation matrices $\{P^{(i)}\}_{i \in [k]}$, and positive real numbers $\{\alpha_i\}_{i \in[k]}$ such that $\sum_{i \in [k]} \alpha_i=1$.  Note that 
	\begin{align*}
	\langle P,AB\T \rangle &= \langle  \sum_{i\in[k]} \alpha_i P^{(i)}, AB\T \rangle=  \sum_{i\in[k]} \alpha_i \langle  P^{(i)}, AB\T \rangle	 \\
	&\leq \sum_{i\in[k]} \alpha_i \langle P', AB\T  \rangle = \langle P', AB\T \rangle \leq \langle P, AB\T \rangle,
	\end{align*}
	% then we have a contradiction, 
	because $P'$ is feasible in rLAP.
	\end{proof}
This relaxation motivates our approach to approximating QAP.

	



\section*{Acknowledgment}

The authors would like to acknowledge two helpful reviewers as well as Lav Varshney for providing the data. This work was partially supported by the Research Program in Applied Neuroscience. 


\bibliography{/Users/jovo/Research/other/latex/library}
\bibliographystyle{IEEEtran}



\end{document}



