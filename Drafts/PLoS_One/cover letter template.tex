\documentclass{letter}
\usepackage{jhuams-ltr}
\usepackage{graphicx}

\setlength{\oddsidemargin}{-.2in} \setlength{\topmargin}{-.55in}
\setlength{\textwidth}{6.9in} \setlength{\textheight}{9.1in}

\signature{Vince Lyzinski \\ Senior Research Scientist\\ Department of Applied Mathematics and Statistics\\ Johns Hopkins University}
\begin{document}
\pagenumbering{gobble}% Remove page numbers (and reset to 1)
\name{Dr.\ Vince Lyzinski}
\position{Senior Research Scientist\\JHU HLT COE}
\address{Stieff Building\\
810 Wyman Park Drive}
%\telephone{(410) 516 7195 }
%\fax{(607) 255 8803}
\email{vlyzins1@jhu.edu}

\begin{letter}
{
Cover letter for the submission of 
\\
``Fast Approximate Quadratic Programming for Large (Brain) Graph Matching''
\\
to PLOS one
}

\date{\today}

\opening{Dear Sir or Madam:}

\bigskip
 
We are delighted to submit to you our manuscript entitled ``\emph{Fast Approximate Quadratic Programming for Large (Brain) Graph Matching}''.
We believe this paper to be an important benchmark in both the quadratic assignment problem and graph matching literatures.  Indeed, this paper has already been cited numerous times and has spawned a bevy of subsequent research papers.  in the literature.  

Quadratic assignment problems have a rich history of theoretic and methodological development.  Yet, the majority of this work has focused on exact algorithms or bounds.  We propose an approach that is quite similar to previous approaches that have been used to obtain exact algorithms or bounds, but we use this approach to obtain an inexact solution. 

Our work was largely inspired by the approximate algorithms 
from Mikhail Zaslavskiy, Francis Bach, and others,
that utilize a ``PATH'' following approach. For that reason, we have made extensive comparisons of our algorithms to PATH based approaches on real world data sets.  The empirical results demonstrate both improved speed and accuracy on over $80\%$ of the benchmarks, as well as a novel real world application of interest: matching brain-graphs (connectomes).

In brief, the basic idea of our algorithm is as follows: we relax the feasible region of the \textbf{NP}-hard quadratic assignment problem to its convex hull, yielding a quadratic problem with linear constraints.  We then successively use the Frank-Wolfe algorithm to descend the gradient.  At completion, we project the final doubly-stochastic matrix to its closest permutation matrix.  All iterations and the final step require $\leq n^3$ operations, like many other approximations.  However, our constants are smaller than those for PATH algorithms because we only descend and project once, versus their multiple descents along the path.  

Thank you for considering publication of our manuscript in your prestigious journal.

Sincerely,\\
\includegraphics[width=5cm]{signature}\\
Vince Lyzinski\\
Senior Research Scientist\\
Human Language Technology Center of Excellence\\
Johns Hopkins University

\end{letter}
\end{document}